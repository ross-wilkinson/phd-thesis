% %TO PRODUCE A STAND-ALONE PDF OF YOUR ABSTRACT, un-comment this header and the \end{document} at the end of the file.
% %
% \documentclass[12pt, a4paper]{memoir}

% \usepackage{mathptmx}
% % *************** Document style definitions ***************

% ******************************************************************
% This file defines the document design.
% Usually it is not necessary to edit this file, but you can use it to change aspects of the design if you want.
% ******************************************************************

%------------------------------------------------------------------------------%
%----------------------------LOAD PACKAGES-------------------------------------%
%------------------------------------------------------------------------------%

% Feel free to alter/add to these packages as you need.
% ******************* Load packages *******************
%Miscellaneous.
%\usepackage{cite}								%Allows abbreviated numerical citations.
\usepackage{amsmath}                            %Allows text in equations
\usepackage{textcomp}                           %Type trademark, copyright symbols
\usepackage[figuresright]{rotating}	            %Allows large tables to be rotated to landscape.
\usepackage{pdfpages}							%Allows you to include full-page pdfs.
\usepackage{wrapfig}							%Lets you wrap text around figures.
\usepackage{subcaption}                         %Lets you include subcaptions within figures
\usepackage[labelfont=bf]{caption}
%Maths stuff.
\usepackage{bm} 								%Bolded maths characters.
\usepackage{upgreek}							%Upright Greek characters.
\usepackage{dsfont}								%Double-struck fonts.
%\usepackage{simplewick}						%For typesetting Wick contractions.
\usepackage{mathtools}						    %Can be used to fine-tune the maths presentation.	
%Text packages.
\usepackage{framed}								%For boxed text.
\usepackage{microtype}						    %pdfLaTeX will fix your kerning.
\usepackage{marvosym}							%Include symbols (like the Euro symbol, etc.).
%Figures.
\usepackage{color}							    %Nice for scalable pdf graphics using InkScape.
\usepackage{transparent}				        %Nice for scalable pdf graphics using InkScape.
\usepackage{placeins}							%Lets you put in a \FloatBarrier to stop figures floating past this command.
\usepackage[T1]{fontenc}
%\usepackage[utf8]{inputenc}
\usepackage{lmodern}
\usepackage[english]{babel}
\usepackage{csquotes}
\usepackage[authordate-trad,backend=biber]{biblatex-chicago}
\usepackage{arydshln} %dashed lines in tabular environment
\usepackage{tabularx} %set fixed table width to wrap text
\usepackage{listings} %insert code with style
\usepackage[framed,numbered,autolinebreaks,useliterate]{mcode}
\usepackage{fourier}

%Lists.
\usepackage{mdframed,mdwlist} 		%Use these for nice lists (less white space).
\usepackage{hyperref}
\hypersetup{colorlinks=false}
\urlstyle{same}

%------------------------------------------------------------------------------%
%---------------------MACROS-----THE-BLACK-------------------------------------%
%------------------------------------------------------------------------------%

%Define a bunch of macros that implement Latin abbreviations.
%COMMENT OUT OR DELETE IF UNDESIRED.
\newcommand{\via}{\textit{via}} %Italicised via.
\newcommand{\ie}{\textit{i.e.}} %Literally.
\newcommand{\eg}{\textit{e.g.}} %For example.
\newcommand{\etc}{\textit{etc.}} %So on...
\newcommand{\vv}{\textit{vice versa}} %And the other way around.
\newcommand{\viz}{\textit{viz}.} %Resulting in.
\newcommand{\cf}{\textit{cf}.} %See, or 'consistent with'.
\newcommand{\apr}{\textit{a priori}} %Before the fact.
\newcommand{\apo}{\textit{a posteriori}} %After the fact.
\newcommand{\vivo}{\textit{in vivo}} %In the flesh.
\newcommand{\situ}{\textit{in situ}} %On location.
\newcommand{\silico}{\textit{in silico}} %Simulation.
\newcommand{\vitro}{\textit{in vitro}} %In glass.
\newcommand{\vs}{\textit{versus}} %James \vs{} Pete.
\newcommand{\ala}{\textit{\`{a} la}} %In the manner of...
\newcommand{\apriori}{\textit{a priori}} %Before hand.
\newcommand{\etal}{\textit{et al.}} %And others, with correct punctuation.
\newcommand{\naive}{na\"\i{}ve} %Queen Amidala is young and \naive{}.
\newcommand{\ra}[1]{\renewcommand{\arraystretch}{#1}}

\newcommand{\chapterendsymbol}{
    \par
    \vspace{\stretch{1}}
    \begin{center}
    \includegraphics[width=60pt]{bicycle.png}
    \end{center}
    \vspace{\stretch{2}}
    }

% *************** End of document style definition ***************

% \begin{document}

% \begin{center}
% 	\textbf{\large Rock and Roll: The Effects of Centre of Mass Movement and Bicycle Lean on the Biomechanics of Cycling}
	
% 	\textbf{Abstract}
	
% 	Ross D. Wilkinson, The University of Queensland, 2020
% \end{center}

%WRITE YOUR ABSTRACT HERE
Cyclists frequently use a non-seated posture when accelerating, climbing steep hills, and sprinting, yet the biomechanical difference between seated and non-seated cycling remains unclear. The purpose of the first study incorporated within this thesis was to test the effects of posture (seated and non-seated) and cadence (70 rpm and 120 rpm) on joint power contributions, effective mechanical advantage, and muscle activity within the lower limb during very-high-power output cycling. Fifteen male participants rode on an instrumented ergometer at 50$\%$ of their individualised instantaneous maximal power (10.74 $\pm$ 1.99 W$\cdot$kg$^{-1}$; above the reported threshold for seated to non-seated transition) in different postures (seated and non-seated) and at different cadences (70 rpm and 120 rpm), whilst lower limb muscle activity, full body motion capture, and crank radial and tangential forces were recorded. A scaled, full-body musculoskeletal model was used to solve inverse kinematics and inverse dynamics to determine joint displacements and net joint moments. Statistical comparisons were made using repeated measure, two-way analyses of variance (posture--cadence). Our results showed significant main effects of posture and cadence on joint power contributions. A key finding was that the non-seated posture increased negative power at the knee, with an associated significant decrease of net power at the knee. The contribution of knee power decreased by 15$\%$ at both 70 and 120 rpm ($\sim$0.8 W$\cdot$kg$^{-1}$) when non-seated compared with seated. Subsequently, hip power and ankle power contributions were significantly higher when non-seated compared with seated at both cadences. In both postures, knee power was 9$\%$ lower at 120 rpm compared with 70 rpm ($\sim$0.4 W$\cdot$kg$^{-1}$). These results evidenced that the contribution of knee joint power to leg power was reduced by switching from a seated to non-seated posture during very-high-power output cycling; however, the size of the reduction is cadence dependent.

Previous research and field observations also suggest that when cyclists ride off the saddle, their centre of mass (CoM) appears to go through a rhythmic vertical oscillation during each crank cycle. Just like in walking and running, the pattern of CoM movement may have a significant impact on the mechanical power that needs to be generated and dissipated by muscle. To date, neither CoM movement strategies during non-seated cycling, nor the limb mechanics that allow this phenomenon to occur have been quantified. In our second study we estimated how much power can be contributed by a rider's body mass at each instant during the crank cycle by combining a kinematic and kinetic approach to measure CoM movement and joint powers of fifteen participants riding in a non-seated posture at three individualised power outputs (10$\%$, 30$\%$, and 50$\%$ of instantaneous maximal power output (P$_{max}$) at two different cadences (70 rpm and 120 rpm). Our analysis revealed that the peak-to-peak amplitude of vertical CoM displacement increased significantly with power output and with decreasing cadence. Accordingly, the greatest peak-to-peak amplitude of CoM displacement (0.06 $\pm$ 0.01 m) and change in total mechanical energy (0.54 $\pm$ 0.12 J$\cdot$kg$^{-1}$) occurred under the combination of high-power output and low cadence. At the same combination of high-power output and low cadence, we found that the peak rate of CoM energy loss (3.87 $\pm$ 0.93 W$\cdot$kg$^{-1}$) was equal to 18$\%$ of the peak instantaneous crank power. Consequently, it appears that for a given power output, changes in CoM energy contribute to peak instantaneous power output at the crank, thus reducing the required muscular contribution. These findings suggest that the rise and fall of a rider's CoM acts as a mechanical amplifier during non-seated cycling,which has important implications for both rider and bicycle performance. 

Building off the results of these first two studies we then investigated the effect of lateral bicycle dynamics on lower limb mechanics and rider CoM mechanical energy fluctuations during non-seated cycling. When riding off the saddle during climbing and sprinting, cyclists appear to coordinate the rhythmic, vertical oscillations of their CoM with the side-to-side lean of the bicycle. Is the coordination of these two motions merely a stability requirement, or could it also be a strategy to more effectively generate crank power? In our third study we again combined a kinematic and kinetic approach, this time to understand how different constraints on bicycle lean influence CoM movement and limb mechanics during non-seated cycling. Ten participants cycled in a non-seated posture at a power output of 5 W$\cdot$kg$^{-1}$ and a cadence of 70 rpm under three bicycle lean conditions: unconstrained on rollers (Unconstrained), under instruction to self-restrict bicycle lean on rollers (Self-Restricted) and constrained in a bicycle trainer (Trainer). Bicycle lean angle in the Unconstrained condition was greater than Self-Restricted and in the Trainer. Vertical CoM displacement, peak vertical crank force, and peak instantaneous crank power in the Unconstrained condition were greater than Self-Restricted but similar to in the Trainer. The amount and rate of energy lost and gained by the rider's CoM in the Unconstrained condition was greater than Self-restricted but similar to in the Trainer. The differences in joint power contributions to total joint power (hip, knee, ankle, and upper body) between conditions were inconclusive. We interpret these results as evidence bicycle lean plays an important role in facilitating the production of high crank force and power output during non-seated cycling by allowing a greater non-muscular contribution to crank power.

In summary, these investigations have established a fundamental but new understanding of the underlying mechanics and energetics of the phenomenon of non-seated cycling, while also pointing towards the potentially detrimental influence of self-restricting bicycle lean when cycling in a non-seated posture at high-power outputs. These findings should be of interest to the field of biomechanics, exercise physiology, and motor control, as well as those involved with optimising rider and bicycle performance.

% \end{document}

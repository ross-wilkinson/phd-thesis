\cleartoevenpage
\pagestyle{empty}	%Use this to suppress the header from the preceding chapter.

\chapter[Discussion and Summary]{Discussion and Summary}
\label{Chap:6}
\pagestyle{headings}

A key aspect of the standing posture is that the rider's hips are higher and further forward in relation to the downstroke pedal \autocite{Caldwell1999}. In our first experiment we attempted to uncover some clues about why cyclists transition from a seated to a non-seated posture when having to generate high torque and power on the cranks. Our analysis revealed that this postural change shifted the period of anti-gravity muscle activity (hip extensors, knee extensors, and ankle plantar flexors) later in the crank cycle. This phase shift of muscle activity redirected the pedal force vector to be more closely aligned with both the knee joint centre and the Earth's gravitational acceleration vector. Together, the more vertically aligned posture and pedal force vector meant that the internal joint moments generated by powerful anti-gravity muscles acted to produce simultaneous propulsion and bodyweight support. These findings highlighted the likelihood of a rider's bodyweight making significant contributions to pedal force and power production during non-seated cycling.

Previous research has provided evidence of a strong positive relationship between a rider's static bodyweight and the peak vertical pedal force produced during non-seated cycling \autocite{Stone1995}. Furthermore, it was shown that peak instantaneous power was dictated primarily by peak vertical pedal force. However, even earlier research had provided indirect evidence that the rider's bodyweight was not static during non-seated cycling, but rather it was raised and lowered during specific phases of the pedal cycle \autocite{Soden1978}. Our second study expanded on these findings by quantifying vertical CoM displacement and the associated changes in total mechanical energy during non-seated cycling at various combinations of cadence and power output. Our analysis confirmed that a rider's CoM gained and lost significant amounts of mechanical energy due to vertical CoM displacement. Furthermore, greater fluctuations in total mechanical energy occurred as power output increased and at lower cadence. It was apparent that the magnitude and phasing of these fluctuations was a deliberate strategy to increase the inertia of the CoM, which facilitated a greater transfer of energy to the crank during the downstroke and then a transfer of energy back to the CoM as the crank passed through bottom dead centre. This intricate flow of energy appeared to benefit the rider by decreasing instantaneous joint power but increasing instantaneous crank power. These findings provided a more detailed insight into how cyclists are able to utilize a non-seated posture and their bodyweight to generate greater levels of pedal force and power output. However, this study was conducted on an ergometer which constrained bicycle lean. Thus, the transferability of these findings to over-ground cycling was unclear.

The use of ergometers in cycling has limited our current understanding of optimal strategies to perform non-seated cycling at high power output because they constrain the lateral dynamics of the bicycle. Capturing rider and bicycle motion over multiple cycles during over-ground cycling would be the ideal scenario, but the calibrated volume of optical motion capture systems limits the number of cycles that can be captured*. Thus, our third study was conducted on a set of electromagnetically braked rollers, which allowed us to analyze multiple cycles of non-seated cycling either with or without restraints on bicycle lean. Our findings showed that vertical CoM displacement, peak vertical crank force, and instantaneous crank power when riders used a preferred amount of bicycle lean was greater than when self-restricting bicycle lean but similar to when in a bicycle trainer. Furthermore, the amount and rate of energy gained and lost by the rider's CoM when riders used a preferred amount of bicycle lean was also greater than when self-restricting bicycle lean but similar to when in a bicycle trainer. First, these results suggest that the findings from our previous studies are likely to transfer to over-ground cycling scenarios. Second, these results suggest that bicycle lean plays an important role in facilitating force and power production by allowing a greater bodyweight and inertia contribution to crank power.

In summary, this thesis has explored some of the fundamental mechanical differences between seated and non-seated cycling and demonstrated the key role that CoM movement might play in the generation of crank power during non-seated cycling. The experiments were devised to explore how joints produce work in the different postures and whether the way in which we perform the movement, or restrictions on the interaction between bicycle and rider, can somehow enhance our ability to produce power (particularly high power). A more detailed summary of the key findings from each study is provided below, along with a commentary on the limitations of each study, topics that require further consideration, and potential practical applications of the evidence presented in this thesis.

\section{Key findings}
\subsection{When cycling at high-power output, switching from a seated to a non-seated posture can reduce net mechanical power requirements at the knee joint.}

The first study in this thesis quantified the difference in net joint power contributions across different joints within the lower limb between seated and non-seated cycling. This adds to existing evidence of differences in net joint moments \autocite{Caldwell1999}, joint kinematics \autocite{Caldwell1999}, muscle activity \autocite{Li1998}, aerobic energy expenditure \autocite{Ryschon1991}, time-to-exhaustion \autocite{Hansen2008}, and maximal power output (\autocite{Millet2002}) between the two postures. Our analysis of lower-limb muscle activity within the same experiment provided evidence that the decrease in net mechanical power at the knee joint was likely due to the action of bi-articular muscles redistributing knee extension power to the hip and ankle. The differences in lower limb muscle activity likely underpinned the redirection of pedal reaction force which lead to greater effective mechanical advantage at the knee joint when non-seated compared to when seated. We interpret these findings as preliminary evidence that the non-seated posture allows both force and power generated from muscular and non-muscular sources to be more effectively transferred across the knee joint and subsequently to the crank. However, the limitations inherent to inverse dynamics and surface EMG must be acknowledged when attempting to infer muscle function. Muscle-level analyses and modelling/simulation studies are required to provide more robust evidence of the function and mechanical work performed by bi-articular muscles during non-seated cycling. 

\subsection{When cycling at high-power output in a non-seated posture, raising and lowering the CoM during specific phases of the crank cycle can amplify crank power.}

The key finding from the second study in this thesis adds to the body of literature on the sources of muscular and non-muscular in cycling by quantifying the total mechanical energy gained and lost by the rider's CoM during non-seated cycling at different power outputs and cadences. Previous research had provided rudimentary evidence that the rider's CoM gains and loses height during each crank cycle, however the associated changes in total mechanical energy of the whole-body had not been shown. Previous research \autocite{Kautz2002} on seated cycling had already shown that total mechanical energy of the legs can be transferred to the crank resulting in external work production. Our results extend this work by showing that the rate of energy transfer between the rider's CoM and the crank could be as high as 4.5 W$\cdot$kg$^{-1}$ during non-seated cycling at high-power output. Our findings also show that the rider's CoM gains and loses greater amounts of energy in response to increasing power output demands and time per crank cycle.  Increasing the total mechanical energy of the CoM as much as possible prior to each downstroke means that there is greater potential during the subsequent downstroke to increase the inertia of the CoM, which can then contribute greater amounts of external work. 

However, not all energy lost by the CoM is necessarily transferred to the crank. It is possible that some of this energy is absorbed by muscle performing negative work or the storage of energy within passive-elastic structures. Any energy stored within lower-limb tendons could be returned later in the crank cycle, which would presumably decrease the metabolic cost of generating force to raise the CoM. The limitations of our analysis mean that we can only speculate whether energy lost by the CoM is partitioned between the crank and elastic structures. Evidence of this mechanism requires further investigation through muscle-level analyses and modelling/simulation studies and remains a goal of future work.

\subsection{When cycling at high-power output in a non-seated posture, leaning the bicycle allows a greater non-muscular contribution to crank force and power.}

The key finding from our third study adds to the body of literature on the influence of bicycle lean on the biomechanics of non-seated cycling by showing that the modification of lateral bicycle dynamics can influence lower limb mechanics and CoM movement. Our results provide evidence that leaning the bicycle allows riders to maximize the non-muscular contribution to crank force and power. To the best of our knowledge, this is the first study to provide a comparison of non-seated cycling under preferred and self-restricted bicycle lean conditions. Bicycle lean occurs due to an imbalance of torque around the bicycle's wheel-ground axis, hence, attempting to restrict bicycle lean requires the rider to reduce the magnitude of these imbalances. Our results show that attempting to do this reduces the amount and rate of CoM energy gained and lost, peak vertical pedal force, and peak instantaneous power output. Thus, our results show that bicycle lean has a significant influence on the temporal nature of force and power production during non-seated cycling on rollers. The limitations of conducting our study on level rollers must be acknowledged and mean there is a possibility that our results may not translate to outdoor cycling conditions. Future investigations comparing preferred and self-restricted bicycle lean conditions during outdoor cycling are warranted. Furthermore, direct evidence pertaining to the effects of bicycle lean on performance could then be provided.    

\section{Future work}

\subsection{Measuring CoM mechanics in ecologically valid conditions.}

It is unclear whether our findings pertaining to CoM movement and the associated changes in mechanical energy can be extrapolated to field conditions. The presence of aerodynamic resistance may alter the preferred movement pattern of the rider because raising and lowering the CoM presumably increases frontal surface area. Thus, measuring CoM movement under ecologically valid conditions would provide important insights into the trade-off between the benefits of non-muscular power contributions and the separate costs of increasing frontal surface area, generating work to raise the CoM, and producing force to support bodyweight. The barrier to conducting this research lies in the difficulty of measuring CoM movement in the field. The typical equipment used to measure CoM movement in laboratory settings (e.g. motion capture, instrumented cranks, and force plates) is likely to be inadequate for high-velocity cycling scenarios or require exorbitant resources. 

Commercial orientation sensors, commonly referred to as Inertial Measurement Units or IMUs, have been proposed as a solution to overcome this barrier \autocite{Pfau2005}. Presumably, this novel approach had not been applied to cycling until now due to two main factors: 1) the difficulty in overcoming integration errors and determining the orientation of the unit, and 2) the lack of evidence regarding the importance of CoM energetics during non-seated cycling. The work presented in this thesis has addressed both of these factors by showing that non-muscular contributions to power output during non-seated cycling are significant and that an inertial sensor placed on the lower back of the rider may be suitable for tracking CoM movement during non-seated cycling (See Appendix \ref{Chap:B}). Analyzing CoM mechanical energy fluctuations during over-ground cycling in a non-seated posture is a future goal, which will require further consideration of changes in altitude. 

\subsection{Contributions of the upper body to power production at the cranks.}

The findings from Chapter \ref{Chap:4} and \ref{Chap:5} suggest a key contribution of the upper body to power production at the crank, but quantifying these contributions and gaining an understanding of the underlying mechanisms requires further investigation. Previous research has provided evidence that creating a pulling force on the handlebars can facilitate a 22$\%$ increase in maximal power output during seated cycling \autocite{Baker2002}. This study compared maximal power output when riders either gripped the handlebar as normal or rested their hands on the handlebar; which prevented them from generating an upward force. 

I plan to expand on these findings by implementing the same experimental comparison during seated and non-seated cycling, while also measuring CoM energetics to explore the underlying mechanisms associated with changes in maximal power output. 

Preliminary data has been collected on 8 participants performing maximal 5-s sprints in a seated and non-seated posture either with or without gripping the handlebar. The outcome of this study will help to explain how the upper body contributes to total joint power and whether it facilitates power output by preventing upward velocity of the CoM during each downstroke. 

\subsection{Non-muscular contributions to instantaneous crank power.}

The findings in Chapter \ref{Chap:4} and \ref{Chap:5}, show that riders can use the inertia of their body mass to amplify instantaneous crank power during sub-maximal non-seated cycling, however, further research is required to confirm the presence and extent of this mechanism during maximal sprints. Based on the presence of this mechanism, I theorize that it may be possible to increase maximal power output during non-seated cycling by adding additional mass to the rider. The muscles of the human body have adapted primarily to help us stand, walk, and run against force due to gravity, however, our potential for maximal power output is often not reached until additional mass is added to the body \autocite{Harris2007}. Previous research on resistance trained rugby-league players performing loaded squat jumps has shown that peak instantaneous power outputs of 4110 $\pm$ 570 W can be achieved by adding loads equivalent to 21.6$\%$ of maximal squat strength ($\sim$57$\%$ of body mass) \autocite{Harris2007}. These results are in line with the theoretical upper limit of human power production \autocite{WILKIE1960a}. 

Preliminary data has been collected on 8 participants performing maximal 5-s sprints in a seated and non-seated posture either with or without added mass. The added mass condition required participants to wear a vest equivalent to 20$\%$ of the participants body mass. Given the short amount of time the rider has to generate enough force to raise and lower the additional mass during each crank cycle, it was estimated that 20$\%$ of body mass would light enough for the rider to achieve the task, yet heavy enough to induce a detectable effect. The outcome of this study may help to explain whether additional mass can facilitate an increase in maximal power output during cycling, however, it is likely that further research will still be required to investigate a spectrum of additional mass conditions. 

\subsection{The influence of bicycle lean and vertical CoM displacement on gross efficiency and maximal power output during non-seated cycling.}

Further research is required to answer the question of whether leaning the bicycle and CoM movement directly affect climbing and sprint performance. The conditions used in Chapter \ref{Chap:4} to investigate the effects of bicycle lean (preferred, self-restricted, and trainer) provided a necessary experimental design to answer this question, however there were some major limitations in the implementation; namely that the rollers we used were restrictive and likely didn't simulate real-world conditions well. To increase the ecological validity of the experiment, it may be more practical to implement the preferred and self-restricted conditions on an inclined treadmill, rather than rollers. To gain insights into climbing performance, expired-gas analysis could be used to measure gross efficiency while cyclists ride in a non-seated posture at an aerobic intensity under the preferred and self-restricted conditions. Bicycle lean and CoM movement could be assessed using either motion capture or IMUs, to confirm the differences in bicycle lean and CoM movement between the conditions and provide further evidence of their relationship. It may also be useful to request that riders change the magnitude of bicycle lean to more directly determine whether there is an optimal strategy. Such a study would l help us understand the optimal bicycle lean and CoM movement strategies during uphill cycling in a non-seated posture.

A similar experimental method could be implemented to understand the influence of bicycle lean and CoM movement on maximal power output during non-seated cycling. The preferred and self-restricted conditions in this case could be conducted in a laboratory or over-ground. The laboratory setting has the potential to provide a more controlled comparison of the preferred and self-restricted conditions to a trainer condition as a single ergometer that can be adjusted to either allow or constrain bicycle lean could be used. I plan to build this device because, to the best of my knowledge, one does not exist. The ergometer would require a hinged platform to allow the ergometer to lean from side to side and a spring-like mechanism to provide a restoring force proportional to the lean angle. This design would provide a suitable replication of the lateral dynamics of a bicycle, which can be reduced to the equations of motion of an underdamped simple harmonic oscillator. This device could be validated by comparing maximal power output to that achieved during over-ground conditions, however, this comparison may be confounded by the difference in aerodynamic resistance between the two settings. Nevertheless, the comparison of preferred and self-restricted conditions in either setting would provide valuable insights into the optimal bicycle lean and CoM movement strategies for achieving maximal power output and maximal velocity while sprinting in a non-seated posture.       

\subsection{The influence of bicycle lean and vertical CoM displacement on muscle function and mechanical work.}

The major limitation of the methodologies used in this thesis is the lack of inference that can be drawn regarding muscle function and muscular work. Inverse dynamic analysis does not provide any indication of the role of individual muscles during movement. Further muscle-level analyses could provide valuable information in this regard, but have yet been applied to non-seated cycling. Previous research has implemented B-mode ultrasound to measure the effects of cadence on muscle fascicle dynamics during constant power seated cycling \autocite{Brennan2019}. This study showed that although cadence did not influence the distribution of lower limb joint power, it did affect fascicle shortening velocities and operating lengths; leading to important differences in muscle efficiency and muscle power capacity. Our results show that CoM mechanical energy fluctuations increase in response to power output and time per crank cycle, which points toward possible changes at a muscular level. This warrants further investigation as changes in muscle efficiency and muscle power capacity likely underlie the preferred movement strategy of riders during non-seated cycling.

\section{Thesis Reflections and Potential Translation of Work}
There are many practical applications of the work in this thesis for rider and bicycle performance. Riders can utilize the knowledge that raising and lowering their CoM while standing may be an optimal strategy for producing maximal impulse and power on the crank. First, finding postures and movement patterns that optimize the trade-off between non-muscular power contribution and aerodynamic drag would allow riders to achieve the highest possible average and peak velocities during climbing and sprints. Second, riders can be confident that swaying the bike and reducing cadence are both beneficial strategies for allowing a greater contribution from non-muscular sources to crank impulse and power. Equipment manufacturers can use this knowledge to understand how shifting from a seated to non-seated posture is likely to effect stresses experienced by the bicycle frame and parts and how shifting the CoM position is likely to effect vehicle dynamics.

The work in this thesis also has a role in better understanding how the body optimizes movement to meet task demands (i.e. increasing gross efficiency or maximal power output). These insights have the potential to contribute to robotic and prosthetic design and help find solutions for neurological conditions where control is hindered.

\clearpage
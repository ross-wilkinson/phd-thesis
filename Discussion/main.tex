\cleartoevenpage
\pagestyle{empty}	%Use this to suppress the header from the preceding chapter.

\chapter[Discussion and Summary]{Discussion and Summary}
\label{Chap:6}
\pagestyle{headings}

As stated by Caldwell et al. (1999), a key aspect of the standing posture is that the rider's hips are higher and further forward in relation to the downstroke pedal. In our first experiment we attempted to uncover some clues about why cyclists transition from a seated to a non-seated posture when having to generate high torque and power on the cranks. Our analysis revealed that this postural change shifted the period of anti-gravity muscle activity (hip extensors, knee extensors, and ankle plantar flexors) later in the crank cycle. This phase shift of muscle activity redirected the pedal force vector to be more closely aligned with both the knee joint centre and the Earth's gravitational acceleration vector. Together, the more vertically aligned posture and pedal force vector meant that the internal joint moments generated by powerful anti-gravity muscles (e.g. gluteus maximus, vastus lateralis, soleus) acted to produce simultaneous propulsion and bodyweight support. These findings highlighted the likelihood of a rider's bodyweight making significant contributions to pedal force and power production during non-seated cycling.

Stone $\&$ Hull (1995) provided evidence of a strong positive relationship between a rider's static bodyweight and the peak vertical pedal force produced during non-seated cycling. Furthermore, they showed that peak instantaneous power was dictated primarily by peak vertical pedal force. However, even earlier research had provided indirect evidence that the rider's bodyweight was not static during non-seated cycling, but rather it was raised and lowered during specific phases of the pedal cycle. Our second study expanded on these findings by quantifying vertical CoM displacement and the associated changes in total mechanical energy during non-seated cycling at various combinations of cadence and power output. Our analysis confirmed that a rider's CoM gained and lost significant amounts of mechanical energy due to vertical CoM displacement. Furthermore, greater fluctuations in total mechanical energy occurred as power output increased and at lower cadence. It was apparent that the magnitude and phasing of these fluctuations was a deliberate strategy to increase the inertia of the CoM, which facilitated a greater transfer of energy to the crank during the downstroke and then a transfer of energy back to the CoM as the crank passed through bottom dead centre. This intricate flow of energy appeared to benefit the rider by decreasing instantaneous joint power but increasing instantaneous crank power. These findings provided a more detailed insight into how cyclists are able to utilise a non-seated posture and their bodyweight to generate greater levels of pedal force and power output. However, this study was conducted on an ergometer which constrained bicycle lean. Thus, the transferability of these findings to over-ground cycling was unclear.

The use of ergometers in cycling has limited our current understanding of optimal strategies to perform non-seated cycling at high power output because they constrain the lateral dynamics of the bicycle (see Chapter \ref{Chap:5}). Capturing rider and bicycle motion over multiple cycles during over-ground cycling would be the ideal scenario, but the calibrated volume of optical motion capture systems limits the number of cycles that can be captured. Thus, our third study was conducted on a set of electromagnetically braked rollers, which allowed us to analyse multiple cycles of non-seated cycling either with or without restraints on bicycle lean. Our findings showed that vertical CoM displacement, peak vertical crank force, and instantaneous crank power when riders used a preferred amount of bicycle lean was greater than when self-restricting bicycle lean but similar to when in a bicycle trainer. Furthermore, the amount and rate of energy gained and lost by the rider's CoM when riders used a preferred amount of bicycle lean was also greater than when self-restricting bicycle lean but similar to when in a bicycle trainer. First, these results suggest that the findings from our previous studies are likely to transfer to over-ground cycling scenarios. Second, these results suggest that bicycle lean plays an important role in facilitating force and power production by allowing a greater bodyweight and inertia contribution to crank power.

In summary, this thesis has explored some of the fundamental mechanical differences between seated and non-seated cycling and demonstrated the key role that bicycle lean CoM movement play in the generation of crank power during non-seated cycling. The experiments were devised to explore how joints produce work in the different postures and whether the way in which we perform the movement, or restrictions on the interaction between bicycle and rider, can somehow enhance our ability to produce power (particularly high power). A further summary of the key findings from each study and how they build on our current understanding of cycling mechanics are provided below, along with a commentary on limitations to the studies, requirement for future research, and potential practical applications of the results.

\section{Key findings}
\subsection{When cycling at high-power output, switching from a seated to a non-seated posture can reduce net mechanical power requirements at the knee joint.}

The study in Chapter \ref{Chap:3} provided the first evidence that the distribution of joint power within the lower limb differs between seated and non-seated cycling. Specifically, the non-seated posture results in a significant reduction in net knee power but an increase in hip and ankle power. Our analysis of lower-limb muscle activity within the same experiment showed that the increase in hip and ankle power was achieved with similar levels of hip extensor and ankle plantar flexor muscle activity; suggesting that the action of bi-articular muscles was responsible for redistributing knee extension power to the hip and ankle. These findings confirm that the transfer of segmental energy within the lower limb is significantly altered by the change in posture, which ties together previous evidence that the change in posture alters net joint moments \autocite{Caldwell1999}, joint kinematics \autocite{Caldwell1999}, and muscle activity \autocite{Li1998}. Furthermore, our results were in agreement with previous findings \autocite{Caldwell1999} that the change in posture alters the direction of the pedal force vector during the downstroke. Our analysis was able to expand on this finding by showing that the redirection of pedal force leads to greater effective mechanical advantage at the knee joint when non-seated compared to when seated. 

Based on this body of evidence, it seems reasonable to suggest that the change in posture also alters the contractile conditions under which muscles generate and dissipate mechanical energy. Gathering evidence of the contractile conditions under which certain muscles produce work \autocite{Brennan2019} would provide important insights pertaining to the effects of posture on aerobic energy expenditure \autocite{Ryschon1991}, time to exhaustion \autocite{Hansen2008}, and maximal power output \autocite{Millet2002}. We interpret our first key finding as preliminary evidence that the non-seated posture allows both force and power generated from muscular and non-muscular sources to be more effectively transferred across the knee joint and subsequently to the crank. However, the limitations inherent to inverse dynamics \autocite{Zajac2002} and surface electromyography \autocite{Enoka2008} must be acknowledged when attempting to infer muscle function. Muscle-level analyses and modelling/simulation studies are required to provide more robust evidence of the function and mechanical work performed by bi-articular muscles during non-seated cycling.

\subsection{When cycling at high-power output in a non-seated posture, raising and lowering the CoM during specific phases of the crank cycle can amplify crank power.}

The second study in this thesis quantified the mechanical energy gained and lost by the rider's CoM during non-seated cycling at different power outputs and cadences. The key finding from this study was that, at high power outputs, the rate of mechanical energy transfer between the rider's CoM and the crank could be as high as 4.5 W$\cdot$kg$^{-1}$. Previous research had provided rudimentary evidence that the rider's CoM gains and loses height during each crank cycle, however, the associated changes in total mechanical energy had not been quantified. Previous research \autocite{Kautz2002} on seated cycling showed that segmental energy of the legs is transferred to the crank resulting in external work production. Our results extend on this work by providing evidence that the amount and rate of energy transfer between the rider's CoM and the crank is much greater during non-seated cycling. Furthermore, it was found that that the rider's CoM gains and loses greater amounts of energy in response to increasing power output and time per crank cycle. This suggests that riders utilise the available time per crank cycle to increase the total mechanical energy of the CoM prior to each downstroke. This increase in CoM height and total mechanical energy provides a greater potential to increase the inertia of the CoM during the downstroke, which can then contribute greater amounts of external work at the crank. 

The results of this study provide impetus to further explore the effects of raising and lowering the CoM on the efficiency of cycling. For instance, not all energy lost by the CoM is necessarily transferred to the crank. It is possible that some of this energy is absorbed by muscle performing negative work or the storage of energy within passive-elastic structures. Any energy stored within lower-limb tendons could be returned later in the crank cycle to lift the rider's CoM at a reduced metabolic cost \autocite{Wilson2011,Uchida2016a,Brennan2018}. The limitations of our analysis mean that we can only speculate whether energy lost by the CoM is partitioned between the crank and elastic structures. Evidence of this mechanism requires further investigation through muscle-level analyses and modelling/simulation studies and remains a goal of future work.

\subsection{When cycling at high-power output in a non-seated posture, leaning the bicycle allows a greater non-muscular contribution to crank force and power.}

The third study in this thesis investigated the effect of constraining bicycle lean on limb mechanics and CoM movement during non-seated cycling. The study showed that self-restricting bicycle lean likely reduced the amount and rate of energy transferred between the rider's CoM and the crank compared to a condition where riders were able to lean the bicycle naturally or when the bicycle was physically constrained in a trainer. This result suggests that leaning the bicycle allows riders to maximize the non-muscular contribution to crank force and power. To the best of our knowledge, this is the first study to provide a comparison of non-seated cycling under preferred and self-restricted bicycle lean conditions. 

Bicycle lean occurs due to an imbalance of torque around the bicycle's wheel-ground axis, hence, attempting to restrict bicycle lean requires the rider to reduce the magnitude of these imbalances. It has been suggested by some authors that leaning the bicycle has a negative effect on non-seated cycling efficiency and performance \autocite{Bouillod2018}. However, our results show that attempting to self-restrict bicycle lean reduces the amount and rate of CoM energy gained and lost, peak vertical pedal force, and peak instantaneous power output. Thus, our results suggest that bicycle lean has a significant positive influence on the temporal nature of force and power production during non-seated cycling on rollers. It is important that future investigations are designed specifically to test the effect of bicycle lean under preferred and self-restricted cycling conditions, rather than making inferences about the effect of bicycle lean when confounding variables are present. 

There are some limitations to the current study that could be addressed in future work. Cycling rollers may not accurately represent bicycle dynamics in a performance environment \autocite{Dressel2012}, which means our results may not translate to treadmill cycling or over-ground cycling. Our study was also conducted on level ground, which may alter the preferred movement pattern of the rider and range of bicycle lean compared to cycling on a sloped treadmill or uphill. Comparing preferred and self-restricted bicycle lean conditions during over-ground cycling is warranted because subtle but important differences in task constraints may affect cycling biomechanics and performance. 

At this time, the scope for collecting biomechanical data during over-ground cycling is limited by equipment. Inertial sensors are a potential solution to this problem \autocite{Pfau2005}, but still require more thorough validation against gold standard measurement systems. Our preliminary investigation of using a single IMU placed on the lower back of the rider to track CoM motion during non-seated cycling showed promising results (See draft manuscript in Appendix \ref{Chap:B}). Our findings suggest that a single IMU can track the orientation and vertical displacement of an attached cluster of reflective markers with high accuracy and precision, however, the single-IMU method overestimates vertical CoM displacement; as this displacement increases, the overestimation error increases linearly. While these results highlight that the motion of body segments other than the torso have a significant effect on whole-body CoM movement, they also show that the discrepancy between IMU and CoM movement increases systematically. Further investigation is required to assess whether linear regression can be used to account for these discrepancies and to better understand the relationship between IMU and actual CoM movement.

\section{Future work}

\subsection{Measuring CoM mechanics in ecologically valid conditions.}

It is unclear whether our findings pertaining to CoM movement and the associated changes in mechanical energy can be extrapolated to field conditions. The presence of aerodynamic resistance may alter the preferred movement pattern of the rider because raising and lowering the CoM presumably changes frontal surface area and the flow of air around the rider. Thus, measuring CoM movement under ecologically valid conditions would provide important insights into the trade-off between the benefits of non-muscular power contributions and the separate costs of aerodynamic resistance, generating work to raise the CoM, and producing force to support bodyweight. The barrier to conducting this research lies in the difficulty of measuring CoM movement in the field. The typical equipment used to measure CoM movement in laboratory settings (e.g. motion capture, instrumented cranks, and force plates) is likely to be inadequate for high-velocity cycling scenarios or require exorbitant resources.

Upon further validation, the single-IMU method investigated in Appendix \ref{Chap:B} could be a used to study the preferred movement pattern of cyclists during over-ground cycling. Other potential solutions for measuring CoM movement during over-ground cycling would be to measure total vertical force produced at the pedals and handlebar or to measure the total vertical force produced on the ground. There are strengths and limitations specific to each of these approaches. The IMU approach is extremely low cost (as little as $\$$10) compared to either instrumenting the pedals and handlebar of a bicycle or using a series of in-ground force plates. Although the in-ground force plates would not require anything to be attached to the rider, a single IMU is unlikely to affect the rider as it is small and lightweight ($\sim$12 grams). Instrumenting the pedals and handlebar of a bicycle would likely require an additional power source and recording device to be mounted to the bicycle. The IMU can also be used on any rider or any bicycle and is unrestricted by location. The instrumented-bicycle and instrumented-surface approaches are likely to be more accurate and precise than the single-IMU approach because they use first principles to solve for vertical CoM acceleration. However, bicycle lean and steering may introduce errors for the instrumented-bicycle approach, unless accelerometers are used in the frame and handlebar to detect the gravity vector. The commercial appeal of integrating a wearable IMU with existing cycling computer technology should also be considered as it may provide an opportunity to collect large datasets on cyclists in any location. The IMU could be used to determine when a cyclist is either seated or non-seated and then separate analysis of cadence and power output between the two postures. It could also be used to gather data on when an individual cyclist typically transitions to a non-seated posture, which could be used as a training dataset for supervised learning algorithms to provide guidance on when that cyclist should transition. A second IMU could also be attached to the bicycle to measure bicycle lean, which would provide insights into the relationship between CoM movement and bicycle lean. Applying this novel approach to understand the importance of bicycle lean and CoM movement on non-seated cycling biomechanics and performance remains a future goal.

\subsection{Contributions of the upper body to power production at the cranks.}

The findings from Chapter \ref{Chap:4} and \ref{Chap:5} suggest a key contribution of the upper body to power production at the crank, but further investigation is required to quantify upper-body power production and understand its relationship to maximal power output during non-seated cycling. Previous research has provided evidence that creating a pulling force on the handlebars can facilitate a 22$\%$ increase in maximal power output during seated cycling \autocite{Baker2002}. This study compared maximal power output when riders either gripped the handlebar as normal or rested their hands on the handlebar; which prevented them from generating an upward force. The limitation of this study is that neither vertical crank force or CoM movement was measured, meaning that it is difficult to infer whether pulling on the handlebar contributes additional power at the crank or merely prevents lower-body power being wasted on lifting the rider's CoM during the downstroke. It is also likely that the effect of pulling on the handlebar will be altered by cadence, power output, and posture. For instance, under low force conditions (i.e. high cadence, low power), there may little need for the arms to resist upward acceleration of the rider's CoM, thus, the contribution of upper-body power to crank power is likely reduced. Riders tend to transition to a non-seated posture under high force scenarios (i.e. low cadence, high power) and when doing so, they support a larger portion of bodyweight at the pedals, align their posture more vertically over the downstroke pedal, and use momentum of their body mass to generate greater peak vertical force and peak instantaneous crank power compared to when seated. Hence, the effect of pulling on the handlebar will depend on pedal force requirements and posture. 

Our work predicts that removing the pulling action of the arms on the handlebar is likely to affect the accelerations and subsequent movement of the CoM. It is hard to predict exactly what might happen to the CoM movement under these circumstances, but the following predictions can be made: 1) if the forces at the cranks remain similar, then the rider's CoM would have to experience more motion, 2) there may be a reduction in crank force and power output because the CoM motion becomes unfavourable (or timing is incorrect), and 3) riders may adopt a different posture to maintain high crank force and power output, but presumably with less non-muscular contributions.

\subsection{Non-muscular contributions to instantaneous crank power.}

The findings in Chapter \ref{Chap:4} and \ref{Chap:5}, show that riders can use momentum of their body mass to amplify instantaneous crank power during sub-maximal non-seated cycling, however, further research is required to confirm the presence and extent of this mechanism during maximal cycling sprints. Based on the presence of this mechanism, I theorise that it may be possible to increase maximal power output during non-seated cycling by adding additional mass to the rider. The muscles of the human body have adapted primarily to help us stand, walk, and run against force due to gravity, however, our potential for maximal power output is often not reached until additional mass is added to the body \autocite{Baker2002,Harris2007}. Previous research on resistance trained rugby-league players performing loaded squat jumps has shown that peak instantaneous power outputs of 4110 $\pm$ 570 W can be achieved by adding loads equivalent to 21.6$\%$ of maximal squat strength ($\sim$57$\%$ of body mass) \autocite{Harris2007}. These results are in line with the theoretical upper limit of human power production \autocite{WILKIE1960a}. 

Our work predicts that some additional torso mass may facilitate an increase in maximal power output during non-seated cycling, but too much additional mass may be detrimental. If a rider's lower-limb muscles have a reserve amount of force generating capacity that is not being utilised under normal body mass conditions, then a certain amount of additional torso mass could enhance their maximal power output. The additional mass could facilitate an increase in maximal power output if it prevents upward accelerations of the rider's CoM during the downstroke or if the rider is able to generate enough force to move the additional mass with the same velocity as during normal body mass conditions. Thus, the optimal amount of additional mass will depend on the force generating capacity of the rider's legs. Too much additional mass may result in the rider being unable to generate enough force to move the mass, which would negate any benefits that CoM momentum may provide. It is also possible that additional mass may not affect maximal power output, which would suggest that the rider's lower-limb muscles are already operating at their maximum force generating capacity. This null effect of additional mass would point toward the pulling action of the arms at the handlebar as a mechanism for ensuring that the maximal force and power generating capacity of the lower-limb muscles is reached. Our understanding of the relationship between body mass, arm action, and maximal power output during non-seated cycling would be greatly enhanced by investigating a rider's maximal power output in response to a spectrum of reduced and additional body mass conditions.

\subsection{The influence of bicycle lean and vertical CoM displacement on gross efficiency and maximal power output during non-seated cycling.}

Further research is required to answer the question of whether leaning the bicycle and CoM movement directly affect climbing and sprint performance. The conditions used in Chapter \ref{Chap:5} to investigate the effects of bicycle lean (preferred, self-restricted, and trainer) provided a necessary experimental design to answer this question, however there were some major limitations in the implementation; namely that the rollers we used were restrictive and likely didn't faithfully simulate real-world conditions. To increase the ecological validity of the experiment, it may be more practical to implement the preferred and self-restricted conditions on an inclined treadmill or during uphill cycling, rather than on rollers. To gain insights into climbing performance, expired-gas analysis could be used to measure gross efficiency while cyclists ride in a non-seated posture at an aerobic intensity during preferred and self-restricted conditions. Bicycle lean and CoM movement could be assessed using either motion capture, crank force, or IMUs, to confirm the differences in bicycle lean and CoM movement between the conditions and provide further evidence of their relationship. It may also be useful to request that riders change the magnitude of bicycle lean to more directly determine whether there is an optimal strategy. Such a study would increase our understanding of the optimal bicycle lean and CoM movement strategies during uphill cycling in a non-seated posture.

A similar experimental paradigm could be implemented to understand the influence of bicycle lean and CoM movement on maximal power output during non-seated cycling. The preferred and self-restricted conditions in this case could be conducted in a laboratory or over ground. The laboratory setting has the potential to provide a more controlled comparison of the preferred and self-restricted conditions to a trainer condition, through the use of a single ergometer which can be adjusted to either allow or constrain bicycle lean. I plan to build this device because, to the best of my knowledge, one does not exist. The ergometer would require a hinged platform to allow the ergometer to lean from side to side and a spring-like mechanism to provide a restoring force proportional to the lean angle. This design would provide a suitable replication of the lateral dynamics of a bicycle, which can be reduced to the equations of motion of an under-damped simple harmonic oscillator. This device could be validated by comparing maximal power output to that achieved during over-ground conditions, however, this comparison may be confounded by the difference in aerodynamic resistance between the two settings. Nevertheless, the comparison of preferred and self-restricted conditions in either a laboratory or outdoor environment would provide valuable insights into the optimal bicycle lean and CoM movement strategies for achieving maximal power output and maximal velocity while sprinting in a non-seated posture.

\subsection{The influence of bicycle lean and vertical CoM displacement on muscle function and mechanical work.}

The major limitation of the methodologies used in this thesis is the lack of inference that can be drawn regarding muscle function and muscular work. Inverse dynamic analysis does not provide any indication of the role of individual muscles during movement due to the action of biarticular muscles and the storage and return of elastic energy \autocite{Zajac2002}. Besides in vivo and in situ techniques, this type of insight regarding the contribution of individual muscles to multi-joint human movements can only be achieved through simulations derived from muscle-based dynamical models of the body. For example, simulations of seated cycling that replicate the kinematics, kinetics, and EMG activity have been used to understand the role of individual muscles in accelerating segments, redistributing segmental energy, and delivering energy to the cranks \autocite{Neptune1998,Martin2018}. These simulation-based studies have provided evidence of causal relationships between the measured kinematics, kinetic, and EMG activity during seated cycling and are therefore the best approach for furthering our understanding of non-seated cycling biomechanics. The challenge in simulating non-seated cycling is the additional degrees of freedom of the movement compared to when seated, however, similar muscle-driven modelling approaches have been successfully applied to complex movements such as walking \autocite{Neptune2004} and running \autocite{Hamner2013}.

Modern imaging techniques, such as magnetic resonance imaging and ultrasonography, have provided unprecedented knowledge on the structure and function of individual muscles and tendons in vivo during complex movements, but have yet been applied to non-seated cycling. Previous research has implemented B-mode ultrasound to measure the effects of workload and cadence on knee-extensor and ankle plantar flexor muscle fascicle dynamics during seated cycling \autocite{Brennan2019,Dick2017}. Brennan et al. (2019) showed that although cadence did not influence the distribution of lower-limb joint power during constant-power seated cycling, it did affect vastus lateralis fascicle shortening velocities and operating lengths; leading to important differences in muscle efficiency and muscle power capacity. Integration of this information with our results showing that CoM mechanical energy fluctuations increase in response to power output and time per crank cycle, provides a reasonable basis to expect that the efficiency and power capacity of individual muscles could be altered by the transition to a non-seated posture during cycling. Investigating the differences in contractile dynamics at an individual muscle level between seated and non-seated cycling remains a future goal, as these differences likely dictate the preferred movement strategy of riders and are extremely relevant for cycling performance.

\section{Potential Translation of Work and Thesis Reflections}
There are many practical applications of the work in this thesis for rider and bicycle performance. Riders can utilise the knowledge that raising and lowering their CoM while standing may be an optimal strategy for producing maximal impulse and power on the crank. First, finding postures and movement patterns that optimise the trade-off between non-muscular power contribution and aerodynamic drag would allow riders to achieve the highest possible average and peak velocities during climbing and sprints. Second, riders can interpret our findings as evidence that leaning the bicycle is a beneficial strategy, or at least not a detrimental strategy, for increasing non-seated cycling performance. Leaning the bicycle allows a greater contribution from non-muscular sources to crank impulse and power. Equipment manufacturers can use this knowledge to understand how shifting from a seated to non-seated posture is likely to affect stresses experienced by different components of the bicycle and how shifting the CoM position is likely to affect vehicle dynamics.

The work in this thesis also has a role in better understanding how the body optimises movement to meet task demands (i.e. increasing gross efficiency or maximal power output). These insights have the potential to contribute to robotic and prosthetic design and help find solutions for neurological conditions where control is hindered. For example, this work has provided valuable information regarding the mechanical requirements of individual joints within the lower limb during non-seated cycling. This information could be used to design orthotic and prosthetic devices that maximise power output during non-seated cycling. Our results have also shown that the mechanical requirements at an individual joint level differ markedly between seated and non-seated cycling, which may provide the impetus for creating orthoses and prostheses that are either specifically designed for non-seated cycling or can be tuned for a given riding posture. 

Reflecting on upon my candidature, I am grateful to have progressed from a know-it-all bicycle shop manager to someone who is willing to admit they know very little about a subject area. As a recreational cyclist, my anecdotal understanding of why I adopted a non-seated posture and leaned the bicycle from side to side was both an opportunity and a threat to my research. At the start of my candidature I was quick to develop and proclaim universal theories of why people transitioned off the saddle and why they leaned the bicycle. Nowadays, these ideas go into a notebook of possible future studies; often being crossed off the list after further critical thought. I have come to recognise that my eagerness to confidently answer problems without sufficient evidence was bred out of insecurity. My initial expectations that completing a PhD would be the hardest thing I have ever done were not wrong. However, it was not my perceived lack of intelligence that was the major source of difficulty, but rather the ability to persevere through the days of confusion, technical problems, and seemingly endless revisions. At first, it can be scary to admit that we know very little about our own research topic, but eventually the ability to admit when we don’t know something becomes empowering. Having the analytical skills to differentiate between evidence and speculation and acknowledging that gathering thorough evidence takes time has helped me to develop confidence and curiosity. I now recognise the complexities of human movement and acknowledge the amount of work that is still required to better understand the control and mechanics of non-seated cycling.

After reviewing cycling literature over the last four years, it is evident that many biomechanical aspects of seated cycling that have been investigated remain unresolved for the non-seated posture. For instance, many cyclists know that the non-seated posture used during uphill climbs is very different from that used during sprints on level terrain. Yet, there is limited evidence regarding preferred movement patterns during non-seated cycling when the nature and motivation of the task vary. The biomechanical difference between these two non-seated techniques could be just as varied, if not more so, than that seen between endurance and time trial seated techniques. The geometry of endurance road bicycles and time trial bicycles are markedly different; thus, it seems reasonable to suggest that bicycles could be designed specifically to enhance either climbing or sprinting performance in a non-seated posture. 

During the initial stages of my candidature, I was often drawn to the intrigue and speculation surrounding the mechanism that may trigger the transition from a seated to a non-seated posture. The intrigue and speculation regarding this topic remain, but one must recognise that it may be extremely difficult to generalise a study's findings outside of the exact conditions tested. In this case, I believe my personal experience of transitioning off the saddle merely due to saddle soreness serves as a helpful reminder that there are many unpredictable factors that may determine a rider's choice of posture. Nevertheless, it may still be useful to use the experimental paradigm of eliciting a transition response to understand why a non-seated posture becomes preferable under different task demands.

I believe that future biomechanical research can play a large role in improving non-seated cycling performance. It is encouraging to see recent evidence showing that cycling equipment which was deemed to have no effect on steady-state seated cycling \autocite{Straw2016} can increase maximal power output in a non-seated posture \autocite{Burns2020}. These findings provide support for revisiting other cycling equipment to understand their effects on maximal power output in a non-seated posture. Thus, many exciting opportunities exist for future work into the biomechanics of non-seated cycling and the effects of bicycle lean, which will not only inform the refinement of cycling training and techniques, but potentially lead to the re-introduction of existing technology and new innovations in equipment design.

A final word on the phenomenon of non-seated cycling: for many of us, cycling in a non-seated posture is instinctive, however, given the relative novelty of the task in terms of the evolution of human movement, it is amazing that the dynamics of the human body and the bicycle can interact so elegantly to generate power output effectively and efficiently.

\clearpage
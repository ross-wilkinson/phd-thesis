
\chapter[Introduction]{Introduction}
\label{Chap:1}

The bicycle is a masterful piece of engineering. The combination of wheels and the additional lever system of the drivetrain decouple the conditions under which muscles generate power, from the forward velocity of the bicycle-rider system. By removing the link between the foot and the ground, a bicycle significantly reduces the cost of transport compared to walking and running and allows the system to coast across the ground, similar to a bird soaring through the air or a fish gliding through water \autocite{Cavagna2017}.

Typically we cycle in a seated posture, whereby the saddle supports much of our bodyweight against the force of gravity. An intriguing aspect of cycling is that, under certain circumstances, riders choose to forego bodyweight support at the saddle and seemingly use their body mass to help produce force on the pedals. Presumably, we learn this technique by cycling in a seated posture under various natural conditions and experiencing perturbations relating to resistance, such as increasing slope and air resistance \autocite{Loeb1995}. Thus, depending on the nature and motivation of the task, we eventually learn to spontaneously transition from a seated to non-seated posture (commonly referred to as standing on the pedals). This transition becomes instinctive, to the point where most of us struggle to explain why we do it. A simple explanation may be that it moves our mass over the pedal, but because producing force to support a larger portion of bodyweight costs energy, we don't always do it \autocite{Kram1990}. So what triggers this transition response in each of us? Perhaps there is a certain level of external torque, or power, or a combination of both that becomes unsustainable when we're seated or just unfavourable compared to a non-seated posture. This topic was addressed in our first study where we uncovered some clues about the biomechanics that underlie the transition response by making people cycle in a seated posture under conditions where they would typically prefer to be non-seated. This study is incorporated as \textbf{Chapter \ref{Chap:3} - The Mechanics of Seated and Non-seated Cycling at Very-High-Power Output: A Joint-level Analysis}.

Once a rider has transitioned to a non-seated posture, they begin to periodically lean the bike from side to side and move their centre of mass (CoM) up and down during each crank cycle. Once again, we may presume that riders learn these movement strategies through the experience of cycling in a non-seated posture under different natural conditions. Similar to the transition response from a seated to non-seated posture, the coordination of CoM movement and bicycle lean seemingly becomes instinctive. Indirect evidence suggests that the magnitude of a rider's vertical CoM displacement and bicycle lean increases at higher power outputs \autocite{Soden1978,Hull1990}. However, the magnitude and pattern of CoM movement has not been directly quantified under different task demands. This gap was addressed in our second study, which is incorporated as \textbf{Chapter \ref{Chap:4} - Riders Use Their Body Mass to Amplify Crank Power during Non-seated Ergometer Cycling}. Based on these findings, there appears to be more to the story than just balance. Perhaps CoM movement and bicycle lean work together to optimise the biomechanics of power production during non-seated cycling. One option to test whether a link exists between these two phenomena and how they may help us to optimise cycling performance would be to compare the CoM movement and limb mechanics of riders cycling in a non-seated posture while different constrains are place on bicycle lean. This was the premise of our third study, which has been incorporated as \textbf{Chapter \ref{Chap:5} - Rock and roll: The influence of bicycle lean on the mechanics of non-seated cycling}.

The scope of this thesis is to answer the general questions of how the biomechanics of seated and non-seated cycling differ under various task demands and whether bicycle lean has a biomechanical effect on how a rider generates power. Thus, a fundamental understanding of the laws that govern lateral bicycle dynamics and how power is generated by the rider is covered briefly within the following literature review.

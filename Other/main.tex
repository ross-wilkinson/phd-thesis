\cleartoevenpage
\pagestyle{empty}	%Use this to suppress the header from the preceding chapter.

\noindent

\chapter[Other work]{Other work}
\label{Chap:Other}	%CREATE YOUR OWN LABEL.

The following chapter is a collection of other works completed during my candidature. Most of this work is either in its preliminary stages or is only loosely related to my thesis topic, which is why it is presented here. Preliminary investigations were completed on the effect of gripping the handlebar on lower body power production, and the effect of adding torso mass on maximal power output.
\clearpage

%%%%%%%%%%%%%%%%%%%%%%%%%%%%%%%%%%%%%%%%%%%%%%%%%%%%%%%%%%%%%
\section{Effect of constraining lateral bicycle dynamics on lower limb muscle activity *TO BE COMPLETED}
%%%%%%%%%%%%%%%%%%%%%%%%%%%%%%%%%%%%%%%%%%%%%%%%%%%%%%%%%%%%%
\pagestyle{headings}

\subsection{Introduction}

\subsection{Methods}

\subsubsection{Active muscle volume}

\subsection{Results}

\begin{figure}[hbtp]
    \centering
    \includegraphics[width=0.9\textwidth]{Other/emg_subject.png}
    \caption[Example data of Root-Mean-Square electromyographic activity of one participant during the Trainer, Preferred, and Self-Restricted conditions.]{\textbf{Example data of Root-Mean-Square electromyographic activity of one participant during the Trainer, Preferred, and Self-Restricted conditions.} Values for each muscle have been normalised to their respective single peak value taken from any cycle across the three conditions. Power output (5 W$\cdot$kg$^{-1}$) and cadence (70 rpm) were the same in each condition.}
    \label{fig:my_label}
\end{figure}

\FloatBarrier

\subsection{Discussion}
TBC

\clearpage
%%%%%%%%%%%%%%%%%%%%%%%%%%%%%%%%%%%%%%%%%%%%%%%%%%%%%%%%%%%%%
\section{Metabolic energy expenditure during seated and non-seated cycling}
%%%%%%%%%%%%%%%%%%%%%%%%%%%%%%%%%%%%%%%%%%%%%%%%%%%%%%%%%%%%%
\pagestyle{headings}

I would like to thank and acknowledge Dr Rodger Kram and Dr Wouter Hoogkamer who at the time were working at the University of Colorado for their significant contribution to the conception, design, collection, and analysis of this study. 

\subsection{Introduction}
The effect of lateral bicycle dynamics on the energetic cost of uphill cycling remains unresolved. Our logic was that cycling on a treadmill should be energetically more costly than cycling in trainer as there should be a cost to maintain dynamic balance of the bicycle-rider system. However, if an interaction effect exists, whereby the additional cost of dynamic balance is less when non-seated than when seated, then this may be evidence that riders are able to utilise lean and steer of the bicycle to somehow offset the cost of balance when cycling in a non-seated posture. Thus, the purpose of this study was to compare the metabolic energy cost of cycling in a seated and non-seated posture either in a ‘constrained’ (trainer) or an ‘unconstrained’ (treadmill) environment.

\subsection{Materials and methods}
\subsubsection{Participants}
Three men (age = 41.0 $\pm$ 14.2 y; height = 1.82 $\pm$ 0.06 m; weight = 72.7 $\pm$ 3.2 kg) volunteered to take part in this study.
\subsubsection{Experimental design}
Participants rode for 5 minutes at a power output of 2.5 W$\cdot$kg$^{-1}$ at a freely chosen cadence in a cycling trainer (constrained condition) and on a treadmill (unconstrained condition) using a non-seated and seated posture. During the cycling trials, participants breathed through a standard mouthpiece (Hans Rudolph 2700, Kansas City, MO, USA) and wore a nose clip, allowing us to measure their rates of oxygen uptake (VO2) and carbon dioxide production (VCO2) with an open-circuit expired gas analysis system (Parvomedics TrueOne 2400, Sandy, UT, USA). Power output, cadence and velocity were measured using a Quark power meter. All metabolic and ventilation measures were averaged over the final two minutes of each trial. A repeated measures, two-way ANOVA was used to test for main and interaction effects (constraint x posture).

\subsection{Results}
The metabolic energy cost of cycling in a non-seated posture (11.64 $\pm$ 2.30) was 8.6$\%$ higher (p$<$05) than seated (10.72 $\pm$ 1.25) during the constrained condition and 10.32$\%$ (11.97 $\pm$ 0.66 vs. 10.85 $\pm$ 0.55, p$<$.05) during the unconstrained condition. Thus, these results provide evidenced that the non-seated posture incurred an additional metabolic energy cost of 3$\%$ in the unconstrained condition compared to when constrained. Interestingly, all three subjects had a significantly higher tidal volume (VT) when cycling in the non-seated posture.

\subsection{Discussion}
This preliminary data supports previous findings that the metabolic energy cost of the non-seated posture is approximately 9$\%$ higher than seated when cycling at moderate aerobic intensity. These results also suggest that a significantly powered study could find that a small increase in metabolic energy cost (approximately 3$\%$) is incurred due to balance when cycling in the non-seated posture, which is not present when seated.

\clearpage
%%%%%%%%%%%%%%%%%%%%%%%%%%%%%%%%%%%%%%%%%%%%%%%%%%%%%%%%%%%%%
\section{A comment on the measurement and interpretation of metabolic efficiency and cost of transport in uphill cycling *TO BE COMPLETED}
%%%%%%%%%%%%%%%%%%%%%%%%%%%%%%%%%%%%%%%%%%%%%%%%%%%%%%%%%%%%%
\pagestyle{headings}

The following is a letter drafted to the Journal of Biomechanics in response to an article (cited below) published earlier this year on the ‘Energy Cost’ and ‘Mechanical Cost’ of uphill cycling. Discussion with colleagues and experts in the field of physiology led us to write this response in order to protect the integrity of the field and the journal in which it was published. The methodological weaknesses present in the study lead the authors to report invalid results and subsequently draw incorrect conclusions from their analysis. We hope the letter will provide impetus for the article to be retracted while also preventing further studies from reporting invalid comparisons of metabolic efficiency and cost of transport during uphill cycling.

\subsection{Comment}
Letter to the Journal of Biomechanics in response to:
Bouillod A, Pinot J, Valade A, Cassirame J, Soto-Romero G, Grappe F. Influence of standing position on mechanical and energy costs in uphill cycling. J Biomech. 2018;72:99–105.

To the editor,

We would like to start by thanking the authors for their efforts in providing a biomechanical comparison of the seated and standing position during uphill cycling. Understanding the effectiveness of the two cycling postures has practical importance for cycling performance and equipment design (1), as well as injury prevention and rehabilitation (2). We are also glad to see attention given to the potential impact of lateral bicycle dynamics on the energy cost and mechanical effectiveness of cycling, for which there is a significant gap in the literature. It is evident that the authors have rare and invaluable access to elite level cyclists and we hope that their scientific endeavours continue into the future.

Our response is focused on pointing out the limited inference that can be drawn from their comparison of metabolic energy expenditure, ‘Energy Cost’ and ‘Mechanical Cost’ between the two cycling positions.  
Of primary concern is the interval period in which oxygen consumption was measured. The period of 30 seconds is far too short to be able to detect meaningful differences in metabolic energy expenditure between the two positions. It is well known that the aerobic system does not stabilise until approximately 3-4 minutes into aerobic physical activity. Thus, protocols lasting a minimum of 5 minutes are typically used. While data obtained during the penultimate and final minute of the task are taken as the most reliable and valid comparison of the true rate of oxygen consumption required for the task. Thus, changing position every 30 seconds invalidates any comparison of ‘Energy Cost’, as the measures of oxygen consumption will be too unstable to assume that the variance is due to the positional change.

Use of the treadmill belt velocity as a denominator within the ‘Energy Cost’ and ‘Mechanical Cost’ equation. Each slope condition required a change in the pitch of the treadmill, which would require a subsequent change in the treadmill belt velocity. These results suggest that at each power output there was an equivalent change in the volume of oxygen consumed and treadmill belt velocity. However, we would assume that slope would have little to no effect on the volume of oxygen consumed when cycling at the same power output. Thus, if our assumptions are true, then the standard deviations of ‘Energy Cost’ reported for each power output in Table 3 cannot be accurate. In figure 1 we present our prediction of the true ‘Energy Cost’ in each condition and the re-calculated mean and standard deviation at each power output level.

The measure of ‘Mechanical Cost’ proposed by the authors will change due to slope. Thus, a better metric would be the discrepancy between external power output at the pedal and the estimated power required by the system to overcome environmental resistance. This would provide a valid comparison regardless of slope as it would account for changes in potential energy due to gravity. For example, attempting to compare the ‘Mechanical Cost’ between each position during a field time trial course, where each position is used intermittently, would be erroneous if there were varying degrees of slope. If, for example, slope was to increase, so too will the ‘Mechanical Cost’ due to the decrease in rear wheel velocity. This would lead to the false assumption that either position, whether seated or standing, is more costly at an 8$\%$ slope compared to a 6$\%$ slope. 

It is likely that constraining cadence between positions but not between the power output or slope conditions would have significantly impacted rider energetics and also the variance assumed to occur from all other variables. It is well-known that the preferred cadence used by cyclists in each position is vastly different (1,3,4), which has a direct impact on metabolic energy expenditure. It has been shown that during standing cycling, lateral bicycle sway away from the mid-line of the bicycle occurs at double the frequency of the cycling cadence adopted by the rider. This dependence of lateral bicycle sway velocity on cadence means that the changes seen in sway velocity between slope and power output conditions could easily be the result of varying cadence. 
An already complex non-parametric, three-way, repeated measures ANOVA (Position x Power x Slope) is further complicated as it does not account for the variance due to changes in cadence between power output and slope conditions. This muddies the reported effects of position, power output, and slope on all measures.

Again, we thank the authors for their contribution, however we believe this paper should be retracted, as the metabolic data is invalid and only adds to the contradictory literature pertaining to the effects of changing position during uphill cycling. \vspace{1em}

Sincerely, \newline 
Ross D. Wilkinson \newline 
\textit{The University of Queensland}


\cleartoevenpage
The following brief communication has been incorporated as Section \ref{sec: chamois} within Chapter \ref{Chap:Other}. 
\textbf{Wilkinson, R.D.}, Marcus, M., Williams, J. and Carver, T. Effect Of chamois design on rider comfort and saddle pressure during sub-maximal cycling. \textit{Annual Meeting of the American-College-of-Sports-Medicine (ACSM)}, Orlando, FL, 28 May-1 June 2019.

\begin{table}[h]
	\begin{center}
	\begin{tabular}{|c|l|l|}
		\hline
		Contributor & Statement of contribution & $\%$ \\
		\hline
		\textbf{Wilkinson, R.D.} & writing of text & 100\\
		& data collection & 5 \\
        & data analysis & 90 \\
		& statistical analysis & 90 \\
		& preparation of figures & 100 \\
		\hline
		Marcus, M. & study design and concept & 20 \\
		& data collection & 25 \\
		\hline
		Williams, J. & study design and concept & 20 \\
		& data collection & 35 \\
		\hline
		Carver, T. & study design and concept & 60 \\
		& data collection & 35 \\
        & data analysis & 10 \\
		& statistical analysis & 10 \\
		& preparation of figures & 10 \\
		& supervision, guidance & 100 \\
		\hline
	\end{tabular}
	\end{center}
\end{table}

\clearpage
\section{Effect of chamois design on rider comfort and saddle pressure during sub-maximal cycling *TO BE COMPLETED}
\label{sec: chamois}
\pagestyle{headings}
\subsection{Abstract}
Here we attempted to quantify saddle comfort during cycling to determine whether a newly designed chamois can make a statistically significant improvement to comfort and whether perceived comfort is correlated to a reduction in peak saddle pressure. In blind tests, participants were asked to rate the comfort of a new `Body Geometry Contour Chamois' and a previous chamois designed by the same company after completing a short bout of seated cycling, while we measured saddle pressure. The aspects of comfort we tested were: overall comfort, genital sensation, genital comfort, sit-bone comfort, buttocks comfort, stability on saddle, off-Saddle comfort. The newly designed chamois significantly reduced peak saddle pressure and increased 'overall comfort' for men and women. However, the relationship between perceived comfort and peak saddle pressure varied greatly within and between subjects. These results provide evidence that chamois design can have a significant effect on peak saddle pressure and perceived comfort during cycling for men and women. 

\subsection{Introduction}
Saddle soreness or discomfort can quickly ruin the enjoyment of cycling and may even be a barrier to further participation in the sport. It has been reported that the contact pressure between the bicycle saddle and sensitive perineal structures may be an underlying mechanism of genital numbness and the development of erectile dysfunction in cyclists \cite{Michiels2015}.  Not surprisingly, the majority of research on the link between saddle pressure, rider comfort and urological disorders has focused primarily on saddle design \cite{Sanford2018}. However, a common oversight is that many riders wear cycling shorts with a built in pad or chamois in order to increase comfort. Given that the cycling pad or chamois is the first point of contact with the skin, we suspect that the type of chamois worn plays an important role in the perceived comfort of riding a bicycle. To date, the effect of chamois design on saddle pressure and perceived comfort in both men and women remains unresolved.

Therefore, the present study aimed to investigate the effects of chamois design on perceived comfort and peak saddle pressure during seated sub-maximal cycling in men and women. We hypothesised that a newly designed chamois would elicit higher comfort ratings than an old chamois design for both men and women and that comfort ratings would be negatively correlated with peak saddle pressure. We also predicted that riders would show a preference for the new chamois design over the old during a standardised week of ‘regular’ training in each chamois. 

\subsection{Methods}
\subsubsection{Participants}
Eighteen recreational riders (9 M; 9 F) were recruited from the surrounding suburbs of Boulder, CO. To be eligible for the study, participants must have been engaged in a minimum of 100 km of riding per week and be specifically training for an event at the time of the study. The testing procedures were explained to the participants before they gave written informed consent. The experimental design of the study was approved by the internal company ethics committee and was carried out in accordance with the general principles outlined within the Declaration of Helsinki. 

\subsubsection{Experimental protocol}
The study design involved both a laboratory- and a field-based protocol. The laboratory protocol required participants to complete a total of four 5-minute cycling trials at 2.5 W$\cdot$kg$^{-1}$ while wearing either a new (A) or old (B) chamois design (2xA; 2xB). Participants completed each trial using their own bicycle and saddle. In order to eliminate bicycle fitting as a confounding factor, the participant’s positioning on the bicycle was compared to a fitting database to ensure that their positioning fell within accepted thresholds for normal rider positioning. Cadence was also kept constant between trials for each participant. During each trial saddle pressure (PSI) was recorded for 30 seconds on a pressure mat (BT-1510, BodiTrack, Vista Medical Ltd, Winnipeg, MB, CAN) with a 45.5 cm x 45.5 cm sensing area placed between the rider’s pelvis and the saddle. At the conclusion of each trial, participants were asked to rate the chamois in seven different comfort categories (Overall Comfort, Genital Sensation, Genital Comfort, Sit Bone Comfort, Buttocks Comfort, Stability on the Saddle, Off Saddle Comfort) on a scale from 0-100. Trials were completed in a randomized order and participants were not told which chamois they were wearing. At the conclusion of the laboratory testing participants were asked to complete one week of regular cycling training in each chamois. The same comfort questionnaire used in laboratory testing was completed at the end of each week. Participants were not told which of the chamois pads corresponded to the new or old design. 

\begin{figure}[htbp]
  \centering
  \begin{subfigure}[b]{0.45\linewidth}
    \includegraphics[width=\linewidth]{Other/chamois_mens.jpg}
     \caption{Mens}
  \end{subfigure}
  \begin{subfigure}[b]{0.45\linewidth}
    \includegraphics[width=\linewidth]{Other/chamois_womens.jpg}
    \caption{Womens}
  \end{subfigure}
  \caption[The new chamois designs were tailored for the characteristic body contours of men and women.]{\textbf{The new chamois designs were tailored for the characteristic body contours of men and women.} }
  \label{fig:Chamois1}
\end{figure}

\subsubsection{Statistical analyses}
A repeated measures, two-way ANOVA was performed to test for main and interaction effects (Chamois x Gender) on saddle pressure and each comfort category in both the laboratory and field study. Within-subject correlations between each comfort rating and saddle pressure were determined for men and women by fitting a repeated measures, mixed regression model.  Significance was set at .05 and data are presented as mean $\pm$ sd.

\subsection{Results}
\subsubsection{Laboratory testing}
Laboratory results showed a significant main effect of chamois design on ‘Overall Comfort’ (F(1,68)=6.1, p=.02). Results from the field testing also showed a significant main effect of chamois design on ‘Overall Comfort’ (F(1,28)=5.0, p=.03) as well as ‘Buttocks Comfort’ (F(1,28)=4.5, p=.04). However, multiple comparisons revealed that Chamois A was only rated significantly higher for ‘Overall Comfort’ by males under laboratory conditions (A=83.3±15.6 vs. B=66.4±23.3, p<.05). Across the group, peak saddle pressures were significantly higher in Chamois B (B=24.5±3.54 vs. A=23.06±3.53, p<.05).

\begin{figure}[htbp]
  \centering
  \begin{subfigure}[b]{0.45\linewidth}
    \includegraphics[width=\linewidth]{Other/chamois_A1.png}
     \caption{New (A)}
  \end{subfigure}
  \begin{subfigure}[b]{0.45\linewidth}
    \includegraphics[width=\linewidth]{Other/chamois_B2.png}
    \caption{Old (B)}
  \end{subfigure}
  \caption[The new (A) chamois design significantly reduced peak saddle pressure for men and women.]{\textbf{The new (A) chamois design significantly reduced peak saddle pressure for men and women.} }
  \label{fig:Chamois2}
\end{figure}

\subsubsection{Field testing}
There was also a significant main effect of Gender on Genital Sensation (F(1,68)=8.2, p=.006) and Genital Comfort (F(1,68)=8.9, p=.004) with females experiencing significantly lower levels of Genital Sensation (F=73.4±23.7 vs. M=89.8±13.3, p<.05) and Genital Comfort (F=71.7±23.6 vs. M=89.0±15.7, p<.05) than males. Linear regression models showed no correlation between any of the comfort criteria ratings and peak saddle pressure. 

\subsection{Discussion}
This study aimed to show that a newly designed chamois would increase perceived comfort and decrease peak saddle pressures during sub-maximal cycling. These results provide evidence for the theory that chamois design is an important factor that affects both peak saddle pressure and perceived comfort for males and females. Yet, it must be noted that perceived comfort and peak saddle pressures are seemingly unrelated. Therefore, it could be suggested that designing a chamois to reduce peak saddle pressure may not be the best approach if the primary purpose is to increase rider comfort. It also appears that perceived comfort is highly individualised and many riders find it difficult to consistently rate comfort of a chamois during sub-maximal trials under laboratory or field testing. The between- and within-subject variability of reported comfort must be considered when designing similar studies in the future.

Further research is required to understand the subtle differences in chamois design that may optimize comfort for men and women and also to address whether there is a link between chamois design and the development of urological disorders due to high cycling workloads.

\subsection{Acknowledgements}
Ross Wilkinson was supported by a Candidate Development Award from The University of Queensland Graduate School to travel to the USA to visit the University of Colorado, Boulder and the Innovation Center run in partnership by Retul\texttrademark and Specialized Bicycle Components, Inc.

\subsection{Conflicts of Interest}
Ross Wilkinson has no conflict of interest relevant to the contents of this article. Mason Marcus, Jason Williams and Todd Carver are employees of Specialized Bicycle Components, Inc.

\chapter{Upper Body Contribution to Maximal Power Output During Cycling}
\label{Chap:D}
\pagestyle{headings}

\section{Abstract}


\section{Introduction}
Previous research has shown that maximal crank power output during seated cycling is reduced by 22$\%$ when riders aren't able to grip the handlebars. The reason for this drop in power is intriguing as the investigation was not able to make any inference about the amount of power generated by lower body or the amount of power generated on, and by, the rider's CoM. Based on our understanding of CoM movement and limb mechanics during cycling, it is possible that the power generated by the lower body was actually the same. It is likely that a portion of the 22$\%$ drop in crank power can be attributed to the absence of power contributed by muscles in the upper body, however an additional portion of this power may have been wasted on producing upward velocity of the CoM during the downstroke. We were interested to know whether this same drop in power would occur when cyclists aren't able to grip the handlebar when in a non-seated posture and whether the lower body is able to produce the same amount of power. Here we combined a kinematic and kinetic approach to compare CoM movement and limb mechanics when cyclists rode with or without gripping the handlebar in a seated and non-seated posture.

\section{Materials and methods}
\subsection{Participants}
Three people (2M; 1F, age = $\pm$, height = $\pm$, mass = $\pm$) volunteered to take part in this study.

\subsection{Experimental design}
Participants began with a 5-min cycling warm-up at 100 W at their preferred cadence. Participants then performed a total of eight 5-s maximal sprints on an instrumented ergometer (Excalibur Sport, Lode BV, Groningen, The Netherlands) in a seated or non-seated posture either with or without gripping the handlebar, while full body motion capture was recorded at 200 Hz (8x Oqus, Qualisys, AB, Sweden) and crank angle, crank radial force, and crank tangential force were recorded at 100 Hz (Axis, Swift Performance, Brisbane, Australia). The ergometer was set to `Isokinetic' mode, which ensured that cadence was kept constant at 120 rpm, which was predicted to elicit near maximal power in each participant \cite{McCartney1983}. A scaled, full-body OpenSim model was used to solve inverse kinematics and inverse dynamics \cite{Rajagopal2016}. Crank length was constant at 175 mm. Participants wore a standardised model of cleated cycling shoe (SH-R070, Shimano, Osaka, Japan) that clipped into the pedals (SH-R540, Shimano, Osaka, Japan).

\begin{figure}[htbp]
  \centering
  \begin{subfigure}[htbp]{0.45\linewidth}
    \includegraphics[width=\linewidth]{Study7/Figure1a.png}
     \caption{Normal hand grip}
  \end{subfigure}
  \begin{subfigure}[htbp]{0.45\linewidth}
    \includegraphics[width=\linewidth]{Study7/Figure1b.png}
    \caption{Without hand grip}
  \end{subfigure}
  \caption[Riders were asked to perform 5-s maximal sprints at 120 rpm in a seated and non-seated posture either with or without gripping the handlebar.]{\textbf{Riders were asked to perform 5-s maximal sprints at 120 rpm in a seated and non-seated posture either with or without gripping the handlebar.} A diagram of the hand position during the with and without hand grip conditions. When riding without hand grip, riders were asked to place their fists on top of the handlebar.}
  \label{fig:grip1}
\end{figure}

\FloatBarrier

\section{Results}
TBC

\section{Discussion}
TBC
